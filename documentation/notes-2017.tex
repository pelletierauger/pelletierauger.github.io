%!TEX root = structure.tex

\newpage
\section{Nouvelle refonte du site Web}

Il y a beaucoup de nouvelles choses à accomplir pour faire de mon site Web un endroit grâce auquel je pourrais partager mon travail de manière optimale.

\begin{enumerate}  
\item Refaire la page d'ouverture, la présentation des projets principaux, en utilisant le nouveau design réalisé récemment.
\item Écrire le contenu des projets.
\item Le plus gros casse-tête : pouvoir inclure des sketchs p5.js dans les pages de mes projets et dans les messages de mon blog.
\item Pouvoir inclure des mathématiques \LaTeX\ avec \textit{MathJax-node} dans les messages de mon blog.
\item Pouvoir inclure du code formatté avec \textit{Highlight.js} dans les messages de mon blog.
\item Écrire une meilleure présentation dans la page \textit{À propos}.
\item Créer une section \textit{Archive} qui présenterait aussi une mappemonde de mes projets, une représentation visuelle de tous les projets que j'ai créés et des connexions entre eux. Il pourrait peut-être même y avoir une place pour les projets non réalisés, pour les \textit{rêves}. Je pourrais définir chacun de mes projets selon certains mots-clés, et selon certaines relations qu'il entretient avec les autres projets, et créer un système où les projets s'organiseraient spatialement eux-mêmes. Ensuite, je dessinerais la position de chaque projet avec une ellipse, et les connexions avec les autres projets par des lignes. La page \textit{Archive} inclurait la mappemonde, une auto-documentation expliquant comment cette mappemonde a été réalisée, et finalement une liste visuellement très sobre de tous les projets. Les points devraient décider eux-mêmes de quelles couleurs ils veulent colorer leur propre partie de la mappemonde. Ainsi, la mappemonde pourrait changer complètement à chaque nouveau projet que j'y ajouterais. La visibilité des projets non réalisés pourrait être optionnelle.
\end{enumerate}